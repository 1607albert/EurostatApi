\documentclass[11pt]{article}

    \usepackage[breakable]{tcolorbox}
    \usepackage{parskip} % Stop auto-indenting (to mimic markdown behaviour)
    
    \usepackage{iftex}
    \ifPDFTeX
    	\usepackage[T1]{fontenc}
    	\usepackage{mathpazo}
    \else
    	\usepackage{fontspec}
    \fi

    % Basic figure setup, for now with no caption control since it's done
    % automatically by Pandoc (which extracts ![](path) syntax from Markdown).
    \usepackage{graphicx}
    % Maintain compatibility with old templates. Remove in nbconvert 6.0
    \let\Oldincludegraphics\includegraphics
    % Ensure that by default, figures have no caption (until we provide a
    % proper Figure object with a Caption API and a way to capture that
    % in the conversion process - todo).
    \usepackage{caption}
    \DeclareCaptionFormat{nocaption}{}
    \captionsetup{format=nocaption,aboveskip=0pt,belowskip=0pt}

    \usepackage{float}
    \floatplacement{figure}{H} % forces figures to be placed at the correct location
    \usepackage{xcolor} % Allow colors to be defined
    \usepackage{enumerate} % Needed for markdown enumerations to work
    \usepackage{geometry} % Used to adjust the document margins
    \usepackage{amsmath} % Equations
    \usepackage{amssymb} % Equations
    \usepackage{textcomp} % defines textquotesingle
    % Hack from http://tex.stackexchange.com/a/47451/13684:
    \AtBeginDocument{%
        \def\PYZsq{\textquotesingle}% Upright quotes in Pygmentized code
    }
    \usepackage{upquote} % Upright quotes for verbatim code
    \usepackage{eurosym} % defines \euro
    \usepackage[mathletters]{ucs} % Extended unicode (utf-8) support
    \usepackage{fancyvrb} % verbatim replacement that allows latex
    \usepackage{grffile} % extends the file name processing of package graphics 
                         % to support a larger range
    \makeatletter % fix for old versions of grffile with XeLaTeX
    \@ifpackagelater{grffile}{2019/11/01}
    {
      % Do nothing on new versions
    }
    {
      \def\Gread@@xetex#1{%
        \IfFileExists{"\Gin@base".bb}%
        {\Gread@eps{\Gin@base.bb}}%
        {\Gread@@xetex@aux#1}%
      }
    }
    \makeatother
    \usepackage[Export]{adjustbox} % Used to constrain images to a maximum size
    \adjustboxset{max size={0.9\linewidth}{0.9\paperheight}}

    % The hyperref package gives us a pdf with properly built
    % internal navigation ('pdf bookmarks' for the table of contents,
    % internal cross-reference links, web links for URLs, etc.)
    \usepackage{hyperref}
    % The default LaTeX title has an obnoxious amount of whitespace. By default,
    % titling removes some of it. It also provides customization options.
    \usepackage{titling}
    \usepackage{longtable} % longtable support required by pandoc >1.10
    \usepackage{booktabs}  % table support for pandoc > 1.12.2
    \usepackage[inline]{enumitem} % IRkernel/repr support (it uses the enumerate* environment)
    \usepackage[normalem]{ulem} % ulem is needed to support strikethroughs (\sout)
                                % normalem makes italics be italics, not underlines
    \usepackage{mathrsfs}
    

    
    % Colors for the hyperref package
    \definecolor{urlcolor}{rgb}{0,.145,.698}
    \definecolor{linkcolor}{rgb}{.71,0.21,0.01}
    \definecolor{citecolor}{rgb}{.12,.54,.11}

    % ANSI colors
    \definecolor{ansi-black}{HTML}{3E424D}
    \definecolor{ansi-black-intense}{HTML}{282C36}
    \definecolor{ansi-red}{HTML}{E75C58}
    \definecolor{ansi-red-intense}{HTML}{B22B31}
    \definecolor{ansi-green}{HTML}{00A250}
    \definecolor{ansi-green-intense}{HTML}{007427}
    \definecolor{ansi-yellow}{HTML}{DDB62B}
    \definecolor{ansi-yellow-intense}{HTML}{B27D12}
    \definecolor{ansi-blue}{HTML}{208FFB}
    \definecolor{ansi-blue-intense}{HTML}{0065CA}
    \definecolor{ansi-magenta}{HTML}{D160C4}
    \definecolor{ansi-magenta-intense}{HTML}{A03196}
    \definecolor{ansi-cyan}{HTML}{60C6C8}
    \definecolor{ansi-cyan-intense}{HTML}{258F8F}
    \definecolor{ansi-white}{HTML}{C5C1B4}
    \definecolor{ansi-white-intense}{HTML}{A1A6B2}
    \definecolor{ansi-default-inverse-fg}{HTML}{FFFFFF}
    \definecolor{ansi-default-inverse-bg}{HTML}{000000}

    % common color for the border for error outputs.
    \definecolor{outerrorbackground}{HTML}{FFDFDF}

    % commands and environments needed by pandoc snippets
    % extracted from the output of `pandoc -s`
    \providecommand{\tightlist}{%
      \setlength{\itemsep}{0pt}\setlength{\parskip}{0pt}}
    \DefineVerbatimEnvironment{Highlighting}{Verbatim}{commandchars=\\\{\}}
    % Add ',fontsize=\small' for more characters per line
    \newenvironment{Shaded}{}{}
    \newcommand{\KeywordTok}[1]{\textcolor[rgb]{0.00,0.44,0.13}{\textbf{{#1}}}}
    \newcommand{\DataTypeTok}[1]{\textcolor[rgb]{0.56,0.13,0.00}{{#1}}}
    \newcommand{\DecValTok}[1]{\textcolor[rgb]{0.25,0.63,0.44}{{#1}}}
    \newcommand{\BaseNTok}[1]{\textcolor[rgb]{0.25,0.63,0.44}{{#1}}}
    \newcommand{\FloatTok}[1]{\textcolor[rgb]{0.25,0.63,0.44}{{#1}}}
    \newcommand{\CharTok}[1]{\textcolor[rgb]{0.25,0.44,0.63}{{#1}}}
    \newcommand{\StringTok}[1]{\textcolor[rgb]{0.25,0.44,0.63}{{#1}}}
    \newcommand{\CommentTok}[1]{\textcolor[rgb]{0.38,0.63,0.69}{\textit{{#1}}}}
    \newcommand{\OtherTok}[1]{\textcolor[rgb]{0.00,0.44,0.13}{{#1}}}
    \newcommand{\AlertTok}[1]{\textcolor[rgb]{1.00,0.00,0.00}{\textbf{{#1}}}}
    \newcommand{\FunctionTok}[1]{\textcolor[rgb]{0.02,0.16,0.49}{{#1}}}
    \newcommand{\RegionMarkerTok}[1]{{#1}}
    \newcommand{\ErrorTok}[1]{\textcolor[rgb]{1.00,0.00,0.00}{\textbf{{#1}}}}
    \newcommand{\NormalTok}[1]{{#1}}
    
    % Additional commands for more recent versions of Pandoc
    \newcommand{\ConstantTok}[1]{\textcolor[rgb]{0.53,0.00,0.00}{{#1}}}
    \newcommand{\SpecialCharTok}[1]{\textcolor[rgb]{0.25,0.44,0.63}{{#1}}}
    \newcommand{\VerbatimStringTok}[1]{\textcolor[rgb]{0.25,0.44,0.63}{{#1}}}
    \newcommand{\SpecialStringTok}[1]{\textcolor[rgb]{0.73,0.40,0.53}{{#1}}}
    \newcommand{\ImportTok}[1]{{#1}}
    \newcommand{\DocumentationTok}[1]{\textcolor[rgb]{0.73,0.13,0.13}{\textit{{#1}}}}
    \newcommand{\AnnotationTok}[1]{\textcolor[rgb]{0.38,0.63,0.69}{\textbf{\textit{{#1}}}}}
    \newcommand{\CommentVarTok}[1]{\textcolor[rgb]{0.38,0.63,0.69}{\textbf{\textit{{#1}}}}}
    \newcommand{\VariableTok}[1]{\textcolor[rgb]{0.10,0.09,0.49}{{#1}}}
    \newcommand{\ControlFlowTok}[1]{\textcolor[rgb]{0.00,0.44,0.13}{\textbf{{#1}}}}
    \newcommand{\OperatorTok}[1]{\textcolor[rgb]{0.40,0.40,0.40}{{#1}}}
    \newcommand{\BuiltInTok}[1]{{#1}}
    \newcommand{\ExtensionTok}[1]{{#1}}
    \newcommand{\PreprocessorTok}[1]{\textcolor[rgb]{0.74,0.48,0.00}{{#1}}}
    \newcommand{\AttributeTok}[1]{\textcolor[rgb]{0.49,0.56,0.16}{{#1}}}
    \newcommand{\InformationTok}[1]{\textcolor[rgb]{0.38,0.63,0.69}{\textbf{\textit{{#1}}}}}
    \newcommand{\WarningTok}[1]{\textcolor[rgb]{0.38,0.63,0.69}{\textbf{\textit{{#1}}}}}
    
    
    % Define a nice break command that doesn't care if a line doesn't already
    % exist.
    \def\br{\hspace*{\fill} \\* }
    % Math Jax compatibility definitions
    \def\gt{>}
    \def\lt{<}
    \let\Oldtex\TeX
    \let\Oldlatex\LaTeX
    \renewcommand{\TeX}{\textrm{\Oldtex}}
    \renewcommand{\LaTeX}{\textrm{\Oldlatex}}
    % Document parameters
    % Document title
    \title{eurostat\_forest\_apiconnect}
    
    
    
    
    
% Pygments definitions
\makeatletter
\def\PY@reset{\let\PY@it=\relax \let\PY@bf=\relax%
    \let\PY@ul=\relax \let\PY@tc=\relax%
    \let\PY@bc=\relax \let\PY@ff=\relax}
\def\PY@tok#1{\csname PY@tok@#1\endcsname}
\def\PY@toks#1+{\ifx\relax#1\empty\else%
    \PY@tok{#1}\expandafter\PY@toks\fi}
\def\PY@do#1{\PY@bc{\PY@tc{\PY@ul{%
    \PY@it{\PY@bf{\PY@ff{#1}}}}}}}
\def\PY#1#2{\PY@reset\PY@toks#1+\relax+\PY@do{#2}}

\@namedef{PY@tok@w}{\def\PY@tc##1{\textcolor[rgb]{0.73,0.73,0.73}{##1}}}
\@namedef{PY@tok@c}{\let\PY@it=\textit\def\PY@tc##1{\textcolor[rgb]{0.25,0.50,0.50}{##1}}}
\@namedef{PY@tok@cp}{\def\PY@tc##1{\textcolor[rgb]{0.74,0.48,0.00}{##1}}}
\@namedef{PY@tok@k}{\let\PY@bf=\textbf\def\PY@tc##1{\textcolor[rgb]{0.00,0.50,0.00}{##1}}}
\@namedef{PY@tok@kp}{\def\PY@tc##1{\textcolor[rgb]{0.00,0.50,0.00}{##1}}}
\@namedef{PY@tok@kt}{\def\PY@tc##1{\textcolor[rgb]{0.69,0.00,0.25}{##1}}}
\@namedef{PY@tok@o}{\def\PY@tc##1{\textcolor[rgb]{0.40,0.40,0.40}{##1}}}
\@namedef{PY@tok@ow}{\let\PY@bf=\textbf\def\PY@tc##1{\textcolor[rgb]{0.67,0.13,1.00}{##1}}}
\@namedef{PY@tok@nb}{\def\PY@tc##1{\textcolor[rgb]{0.00,0.50,0.00}{##1}}}
\@namedef{PY@tok@nf}{\def\PY@tc##1{\textcolor[rgb]{0.00,0.00,1.00}{##1}}}
\@namedef{PY@tok@nc}{\let\PY@bf=\textbf\def\PY@tc##1{\textcolor[rgb]{0.00,0.00,1.00}{##1}}}
\@namedef{PY@tok@nn}{\let\PY@bf=\textbf\def\PY@tc##1{\textcolor[rgb]{0.00,0.00,1.00}{##1}}}
\@namedef{PY@tok@ne}{\let\PY@bf=\textbf\def\PY@tc##1{\textcolor[rgb]{0.82,0.25,0.23}{##1}}}
\@namedef{PY@tok@nv}{\def\PY@tc##1{\textcolor[rgb]{0.10,0.09,0.49}{##1}}}
\@namedef{PY@tok@no}{\def\PY@tc##1{\textcolor[rgb]{0.53,0.00,0.00}{##1}}}
\@namedef{PY@tok@nl}{\def\PY@tc##1{\textcolor[rgb]{0.63,0.63,0.00}{##1}}}
\@namedef{PY@tok@ni}{\let\PY@bf=\textbf\def\PY@tc##1{\textcolor[rgb]{0.60,0.60,0.60}{##1}}}
\@namedef{PY@tok@na}{\def\PY@tc##1{\textcolor[rgb]{0.49,0.56,0.16}{##1}}}
\@namedef{PY@tok@nt}{\let\PY@bf=\textbf\def\PY@tc##1{\textcolor[rgb]{0.00,0.50,0.00}{##1}}}
\@namedef{PY@tok@nd}{\def\PY@tc##1{\textcolor[rgb]{0.67,0.13,1.00}{##1}}}
\@namedef{PY@tok@s}{\def\PY@tc##1{\textcolor[rgb]{0.73,0.13,0.13}{##1}}}
\@namedef{PY@tok@sd}{\let\PY@it=\textit\def\PY@tc##1{\textcolor[rgb]{0.73,0.13,0.13}{##1}}}
\@namedef{PY@tok@si}{\let\PY@bf=\textbf\def\PY@tc##1{\textcolor[rgb]{0.73,0.40,0.53}{##1}}}
\@namedef{PY@tok@se}{\let\PY@bf=\textbf\def\PY@tc##1{\textcolor[rgb]{0.73,0.40,0.13}{##1}}}
\@namedef{PY@tok@sr}{\def\PY@tc##1{\textcolor[rgb]{0.73,0.40,0.53}{##1}}}
\@namedef{PY@tok@ss}{\def\PY@tc##1{\textcolor[rgb]{0.10,0.09,0.49}{##1}}}
\@namedef{PY@tok@sx}{\def\PY@tc##1{\textcolor[rgb]{0.00,0.50,0.00}{##1}}}
\@namedef{PY@tok@m}{\def\PY@tc##1{\textcolor[rgb]{0.40,0.40,0.40}{##1}}}
\@namedef{PY@tok@gh}{\let\PY@bf=\textbf\def\PY@tc##1{\textcolor[rgb]{0.00,0.00,0.50}{##1}}}
\@namedef{PY@tok@gu}{\let\PY@bf=\textbf\def\PY@tc##1{\textcolor[rgb]{0.50,0.00,0.50}{##1}}}
\@namedef{PY@tok@gd}{\def\PY@tc##1{\textcolor[rgb]{0.63,0.00,0.00}{##1}}}
\@namedef{PY@tok@gi}{\def\PY@tc##1{\textcolor[rgb]{0.00,0.63,0.00}{##1}}}
\@namedef{PY@tok@gr}{\def\PY@tc##1{\textcolor[rgb]{1.00,0.00,0.00}{##1}}}
\@namedef{PY@tok@ge}{\let\PY@it=\textit}
\@namedef{PY@tok@gs}{\let\PY@bf=\textbf}
\@namedef{PY@tok@gp}{\let\PY@bf=\textbf\def\PY@tc##1{\textcolor[rgb]{0.00,0.00,0.50}{##1}}}
\@namedef{PY@tok@go}{\def\PY@tc##1{\textcolor[rgb]{0.53,0.53,0.53}{##1}}}
\@namedef{PY@tok@gt}{\def\PY@tc##1{\textcolor[rgb]{0.00,0.27,0.87}{##1}}}
\@namedef{PY@tok@err}{\def\PY@bc##1{{\setlength{\fboxsep}{\string -\fboxrule}\fcolorbox[rgb]{1.00,0.00,0.00}{1,1,1}{\strut ##1}}}}
\@namedef{PY@tok@kc}{\let\PY@bf=\textbf\def\PY@tc##1{\textcolor[rgb]{0.00,0.50,0.00}{##1}}}
\@namedef{PY@tok@kd}{\let\PY@bf=\textbf\def\PY@tc##1{\textcolor[rgb]{0.00,0.50,0.00}{##1}}}
\@namedef{PY@tok@kn}{\let\PY@bf=\textbf\def\PY@tc##1{\textcolor[rgb]{0.00,0.50,0.00}{##1}}}
\@namedef{PY@tok@kr}{\let\PY@bf=\textbf\def\PY@tc##1{\textcolor[rgb]{0.00,0.50,0.00}{##1}}}
\@namedef{PY@tok@bp}{\def\PY@tc##1{\textcolor[rgb]{0.00,0.50,0.00}{##1}}}
\@namedef{PY@tok@fm}{\def\PY@tc##1{\textcolor[rgb]{0.00,0.00,1.00}{##1}}}
\@namedef{PY@tok@vc}{\def\PY@tc##1{\textcolor[rgb]{0.10,0.09,0.49}{##1}}}
\@namedef{PY@tok@vg}{\def\PY@tc##1{\textcolor[rgb]{0.10,0.09,0.49}{##1}}}
\@namedef{PY@tok@vi}{\def\PY@tc##1{\textcolor[rgb]{0.10,0.09,0.49}{##1}}}
\@namedef{PY@tok@vm}{\def\PY@tc##1{\textcolor[rgb]{0.10,0.09,0.49}{##1}}}
\@namedef{PY@tok@sa}{\def\PY@tc##1{\textcolor[rgb]{0.73,0.13,0.13}{##1}}}
\@namedef{PY@tok@sb}{\def\PY@tc##1{\textcolor[rgb]{0.73,0.13,0.13}{##1}}}
\@namedef{PY@tok@sc}{\def\PY@tc##1{\textcolor[rgb]{0.73,0.13,0.13}{##1}}}
\@namedef{PY@tok@dl}{\def\PY@tc##1{\textcolor[rgb]{0.73,0.13,0.13}{##1}}}
\@namedef{PY@tok@s2}{\def\PY@tc##1{\textcolor[rgb]{0.73,0.13,0.13}{##1}}}
\@namedef{PY@tok@sh}{\def\PY@tc##1{\textcolor[rgb]{0.73,0.13,0.13}{##1}}}
\@namedef{PY@tok@s1}{\def\PY@tc##1{\textcolor[rgb]{0.73,0.13,0.13}{##1}}}
\@namedef{PY@tok@mb}{\def\PY@tc##1{\textcolor[rgb]{0.40,0.40,0.40}{##1}}}
\@namedef{PY@tok@mf}{\def\PY@tc##1{\textcolor[rgb]{0.40,0.40,0.40}{##1}}}
\@namedef{PY@tok@mh}{\def\PY@tc##1{\textcolor[rgb]{0.40,0.40,0.40}{##1}}}
\@namedef{PY@tok@mi}{\def\PY@tc##1{\textcolor[rgb]{0.40,0.40,0.40}{##1}}}
\@namedef{PY@tok@il}{\def\PY@tc##1{\textcolor[rgb]{0.40,0.40,0.40}{##1}}}
\@namedef{PY@tok@mo}{\def\PY@tc##1{\textcolor[rgb]{0.40,0.40,0.40}{##1}}}
\@namedef{PY@tok@ch}{\let\PY@it=\textit\def\PY@tc##1{\textcolor[rgb]{0.25,0.50,0.50}{##1}}}
\@namedef{PY@tok@cm}{\let\PY@it=\textit\def\PY@tc##1{\textcolor[rgb]{0.25,0.50,0.50}{##1}}}
\@namedef{PY@tok@cpf}{\let\PY@it=\textit\def\PY@tc##1{\textcolor[rgb]{0.25,0.50,0.50}{##1}}}
\@namedef{PY@tok@c1}{\let\PY@it=\textit\def\PY@tc##1{\textcolor[rgb]{0.25,0.50,0.50}{##1}}}
\@namedef{PY@tok@cs}{\let\PY@it=\textit\def\PY@tc##1{\textcolor[rgb]{0.25,0.50,0.50}{##1}}}

\def\PYZbs{\char`\\}
\def\PYZus{\char`\_}
\def\PYZob{\char`\{}
\def\PYZcb{\char`\}}
\def\PYZca{\char`\^}
\def\PYZam{\char`\&}
\def\PYZlt{\char`\<}
\def\PYZgt{\char`\>}
\def\PYZsh{\char`\#}
\def\PYZpc{\char`\%}
\def\PYZdl{\char`\$}
\def\PYZhy{\char`\-}
\def\PYZsq{\char`\'}
\def\PYZdq{\char`\"}
\def\PYZti{\char`\~}
% for compatibility with earlier versions
\def\PYZat{@}
\def\PYZlb{[}
\def\PYZrb{]}
\makeatother


    % For linebreaks inside Verbatim environment from package fancyvrb. 
    \makeatletter
        \newbox\Wrappedcontinuationbox 
        \newbox\Wrappedvisiblespacebox 
        \newcommand*\Wrappedvisiblespace {\textcolor{red}{\textvisiblespace}} 
        \newcommand*\Wrappedcontinuationsymbol {\textcolor{red}{\llap{\tiny$\m@th\hookrightarrow$}}} 
        \newcommand*\Wrappedcontinuationindent {3ex } 
        \newcommand*\Wrappedafterbreak {\kern\Wrappedcontinuationindent\copy\Wrappedcontinuationbox} 
        % Take advantage of the already applied Pygments mark-up to insert 
        % potential linebreaks for TeX processing. 
        %        {, <, #, %, $, ' and ": go to next line. 
        %        _, }, ^, &, >, - and ~: stay at end of broken line. 
        % Use of \textquotesingle for straight quote. 
        \newcommand*\Wrappedbreaksatspecials {% 
            \def\PYGZus{\discretionary{\char`\_}{\Wrappedafterbreak}{\char`\_}}% 
            \def\PYGZob{\discretionary{}{\Wrappedafterbreak\char`\{}{\char`\{}}% 
            \def\PYGZcb{\discretionary{\char`\}}{\Wrappedafterbreak}{\char`\}}}% 
            \def\PYGZca{\discretionary{\char`\^}{\Wrappedafterbreak}{\char`\^}}% 
            \def\PYGZam{\discretionary{\char`\&}{\Wrappedafterbreak}{\char`\&}}% 
            \def\PYGZlt{\discretionary{}{\Wrappedafterbreak\char`\<}{\char`\<}}% 
            \def\PYGZgt{\discretionary{\char`\>}{\Wrappedafterbreak}{\char`\>}}% 
            \def\PYGZsh{\discretionary{}{\Wrappedafterbreak\char`\#}{\char`\#}}% 
            \def\PYGZpc{\discretionary{}{\Wrappedafterbreak\char`\%}{\char`\%}}% 
            \def\PYGZdl{\discretionary{}{\Wrappedafterbreak\char`\$}{\char`\$}}% 
            \def\PYGZhy{\discretionary{\char`\-}{\Wrappedafterbreak}{\char`\-}}% 
            \def\PYGZsq{\discretionary{}{\Wrappedafterbreak\textquotesingle}{\textquotesingle}}% 
            \def\PYGZdq{\discretionary{}{\Wrappedafterbreak\char`\"}{\char`\"}}% 
            \def\PYGZti{\discretionary{\char`\~}{\Wrappedafterbreak}{\char`\~}}% 
        } 
        % Some characters . , ; ? ! / are not pygmentized. 
        % This macro makes them "active" and they will insert potential linebreaks 
        \newcommand*\Wrappedbreaksatpunct {% 
            \lccode`\~`\.\lowercase{\def~}{\discretionary{\hbox{\char`\.}}{\Wrappedafterbreak}{\hbox{\char`\.}}}% 
            \lccode`\~`\,\lowercase{\def~}{\discretionary{\hbox{\char`\,}}{\Wrappedafterbreak}{\hbox{\char`\,}}}% 
            \lccode`\~`\;\lowercase{\def~}{\discretionary{\hbox{\char`\;}}{\Wrappedafterbreak}{\hbox{\char`\;}}}% 
            \lccode`\~`\:\lowercase{\def~}{\discretionary{\hbox{\char`\:}}{\Wrappedafterbreak}{\hbox{\char`\:}}}% 
            \lccode`\~`\?\lowercase{\def~}{\discretionary{\hbox{\char`\?}}{\Wrappedafterbreak}{\hbox{\char`\?}}}% 
            \lccode`\~`\!\lowercase{\def~}{\discretionary{\hbox{\char`\!}}{\Wrappedafterbreak}{\hbox{\char`\!}}}% 
            \lccode`\~`\/\lowercase{\def~}{\discretionary{\hbox{\char`\/}}{\Wrappedafterbreak}{\hbox{\char`\/}}}% 
            \catcode`\.\active
            \catcode`\,\active 
            \catcode`\;\active
            \catcode`\:\active
            \catcode`\?\active
            \catcode`\!\active
            \catcode`\/\active 
            \lccode`\~`\~ 	
        }
    \makeatother

    \let\OriginalVerbatim=\Verbatim
    \makeatletter
    \renewcommand{\Verbatim}[1][1]{%
        %\parskip\z@skip
        \sbox\Wrappedcontinuationbox {\Wrappedcontinuationsymbol}%
        \sbox\Wrappedvisiblespacebox {\FV@SetupFont\Wrappedvisiblespace}%
        \def\FancyVerbFormatLine ##1{\hsize\linewidth
            \vtop{\raggedright\hyphenpenalty\z@\exhyphenpenalty\z@
                \doublehyphendemerits\z@\finalhyphendemerits\z@
                \strut ##1\strut}%
        }%
        % If the linebreak is at a space, the latter will be displayed as visible
        % space at end of first line, and a continuation symbol starts next line.
        % Stretch/shrink are however usually zero for typewriter font.
        \def\FV@Space {%
            \nobreak\hskip\z@ plus\fontdimen3\font minus\fontdimen4\font
            \discretionary{\copy\Wrappedvisiblespacebox}{\Wrappedafterbreak}
            {\kern\fontdimen2\font}%
        }%
        
        % Allow breaks at special characters using \PYG... macros.
        \Wrappedbreaksatspecials
        % Breaks at punctuation characters . , ; ? ! and / need catcode=\active 	
        \OriginalVerbatim[#1,codes*=\Wrappedbreaksatpunct]%
    }
    \makeatother

    % Exact colors from NB
    \definecolor{incolor}{HTML}{303F9F}
    \definecolor{outcolor}{HTML}{D84315}
    \definecolor{cellborder}{HTML}{CFCFCF}
    \definecolor{cellbackground}{HTML}{F7F7F7}
    
    % prompt
    \makeatletter
    \newcommand{\boxspacing}{\kern\kvtcb@left@rule\kern\kvtcb@boxsep}
    \makeatother
    \newcommand{\prompt}[4]{
        {\ttfamily\llap{{\color{#2}[#3]:\hspace{3pt}#4}}\vspace{-\baselineskip}}
    }
    

    
    % Prevent overflowing lines due to hard-to-break entities
    \sloppy 
    % Setup hyperref package
    \hypersetup{
      breaklinks=true,  % so long urls are correctly broken across lines
      colorlinks=true,
      urlcolor=urlcolor,
      linkcolor=linkcolor,
      citecolor=citecolor,
      }
    % Slightly bigger margins than the latex defaults
    
    \geometry{verbose,tmargin=1in,bmargin=1in,lmargin=1in,rmargin=1in}
    
    

\begin{document}
    
    \maketitle
    
    

    
    \begin{tcolorbox}[breakable, size=fbox, boxrule=1pt, pad at break*=1mm,colback=cellbackground, colframe=cellborder]
\prompt{In}{incolor}{1}{\boxspacing}
\begin{Verbatim}[commandchars=\\\{\}]
\PY{k+kn}{import} \PY{n+nn}{requests}
\end{Verbatim}
\end{tcolorbox}

    \begin{tcolorbox}[breakable, size=fbox, boxrule=1pt, pad at break*=1mm,colback=cellbackground, colframe=cellborder]
\prompt{In}{incolor}{2}{\boxspacing}
\begin{Verbatim}[commandchars=\\\{\}]
\PY{c+c1}{\PYZsh{} Eurostat API connect and query generation using the Query generator from Eurostat}
\PY{n}{eurostat\PYZus{}url\PYZus{}service} \PY{o}{=} \PY{l+s+s1}{\PYZsq{}}\PY{l+s+s1}{http://ec.europa.eu/eurostat/wdds/rest/data/v2.1/json/en/}\PY{l+s+s1}{\PYZsq{}}

\PY{n}{queryEurostat} \PY{o}{=} \PY{l+s+s2}{\PYZdq{}}\PY{l+s+s2}{for\PYZus{}area?groupedIndicators=1\PYZam{}filterNonGeo=1\PYZam{}precision=1\PYZam{}indic\PYZus{}fo=FOR\PYZam{}unit=THS\PYZus{}HA\PYZam{}time=1990\PYZam{}time=2000\PYZam{}time=2010\PYZam{}time=2020}\PY{l+s+s2}{\PYZdq{}}
\PY{n}{rest\PYZus{}petition} \PY{o}{=} \PY{n}{eurostat\PYZus{}url\PYZus{}service} \PY{o}{+} \PY{n}{queryEurostat}
\PY{n}{response} \PY{o}{=} \PY{n}{requests}\PY{o}{.}\PY{n}{get}\PY{p}{(}\PY{n}{rest\PYZus{}petition}\PY{p}{)}

\PY{c+c1}{\PYZsh{} Print the Query in the URL}
\PY{n}{rest\PYZus{}petition}
\end{Verbatim}
\end{tcolorbox}

            \begin{tcolorbox}[breakable, size=fbox, boxrule=.5pt, pad at break*=1mm, opacityfill=0]
\prompt{Out}{outcolor}{2}{\boxspacing}
\begin{Verbatim}[commandchars=\\\{\}]
'http://ec.europa.eu/eurostat/wdds/rest/data/v2.1/json/en/for\_area?groupedIndica
tors=1\&filterNonGeo=1\&precision=1\&indic\_fo=FOR\&unit=THS\_HA\&time=1990\&time=2000\&t
ime=2010\&time=2020'
\end{Verbatim}
\end{tcolorbox}
        
    \begin{tcolorbox}[breakable, size=fbox, boxrule=1pt, pad at break*=1mm,colback=cellbackground, colframe=cellborder]
\prompt{In}{incolor}{3}{\boxspacing}
\begin{Verbatim}[commandchars=\\\{\}]
\PY{c+c1}{\PYZsh{} As we know that the answer will be in JSON format..}

\PY{k+kn}{import} \PY{n+nn}{json}

\PY{n}{jsonEurostat} \PY{o}{=} \PY{n}{json}\PY{o}{.}\PY{n}{loads}\PY{p}{(}\PY{n}{response}\PY{o}{.}\PY{n}{text}\PY{p}{)}
\PY{c+c1}{\PYZsh{} jsonEurostat}
\PY{c+c1}{\PYZsh{} the API response looks like that, in this case in the initial properties, we can see the \PYZdq{}label\PYZdq{} }
\PY{c+c1}{\PYZsh{} title declaring = \PYZsq{}Area of wooded land (source: FAO \PYZhy{} FE)\PYZsq{}}
\PY{c+c1}{\PYZsh{} notice also the update date in order to check that the service is constantly updated.}
\end{Verbatim}
\end{tcolorbox}

    \begin{tcolorbox}[breakable, size=fbox, boxrule=1pt, pad at break*=1mm,colback=cellbackground, colframe=cellborder]
\prompt{In}{incolor}{4}{\boxspacing}
\begin{Verbatim}[commandchars=\\\{\}]
\PY{k+kn}{from} \PY{n+nn}{pycountry\PYZus{}convert} \PY{k+kn}{import} \PY{n}{country\PYZus{}alpha2\PYZus{}to\PYZus{}continent\PYZus{}code}\PY{p}{,} \PY{n}{country\PYZus{}name\PYZus{}to\PYZus{}country\PYZus{}alpha2}
\PY{k+kn}{from} \PY{n+nn}{geopy}\PY{n+nn}{.}\PY{n+nn}{geocoders} \PY{k+kn}{import} \PY{n}{Nominatim}
\PY{n}{geolocator} \PY{o}{=} \PY{n}{Nominatim}\PY{p}{(}\PY{n}{user\PYZus{}agent}\PY{o}{=}\PY{l+s+s2}{\PYZdq{}}\PY{l+s+s2}{App\PYZus{}prueba}\PY{l+s+s2}{\PYZdq{}}\PY{p}{)}

\PY{c+c1}{\PYZsh{} As we are looking to represent the data in a map we will have to use a geocoding system to}
\PY{c+c1}{\PYZsh{} get and recognize the country codes so we import geopy.geocoders and pycountry\PYZus{}convert}

\PY{c+c1}{\PYZsh{} By other hand we must know that Eurostat give the data by countries in order, }
\PY{c+c1}{\PYZsh{} so as we asked for 4 different year data collections we first must \PYZdq{}group\PYZdq{} the records}
\PY{c+c1}{\PYZsh{} by country and year}

\PY{n}{countries} \PY{o}{=} \PY{p}{\PYZob{}}\PY{p}{\PYZcb{}} \PY{c+c1}{\PYZsh{}for now we would work with the data as a JSON not as an array}
\PY{c+c1}{\PYZsh{} lets use the JSON working format to get to the values we are searching for,}
\PY{c+c1}{\PYZsh{} in this case we want to acces to \PYZdq{}dimension/geo/category/label\PYZdq{} in order to get the}
\PY{c+c1}{\PYZsh{} relation between the unique country name and the country codes(primary keys)}
\PY{c+c1}{\PYZsh{} We select also tha country index if the primary key is the same as the country index }
\PY{c+c1}{\PYZsh{} we establish the relation}
\PY{k}{for} \PY{n}{label} \PY{o+ow}{in} \PY{n}{jsonEurostat}\PY{p}{[}\PY{l+s+s1}{\PYZsq{}}\PY{l+s+s1}{dimension}\PY{l+s+s1}{\PYZsq{}}\PY{p}{]}\PY{p}{[}\PY{l+s+s1}{\PYZsq{}}\PY{l+s+s1}{geo}\PY{l+s+s1}{\PYZsq{}}\PY{p}{]}\PY{p}{[}\PY{l+s+s1}{\PYZsq{}}\PY{l+s+s1}{category}\PY{l+s+s1}{\PYZsq{}}\PY{p}{]}\PY{p}{[}\PY{l+s+s1}{\PYZsq{}}\PY{l+s+s1}{label}\PY{l+s+s1}{\PYZsq{}}\PY{p}{]}\PY{p}{:}
    \PY{k}{for} \PY{n}{index} \PY{o+ow}{in} \PY{n}{jsonEurostat}\PY{p}{[}\PY{l+s+s1}{\PYZsq{}}\PY{l+s+s1}{dimension}\PY{l+s+s1}{\PYZsq{}}\PY{p}{]}\PY{p}{[}\PY{l+s+s1}{\PYZsq{}}\PY{l+s+s1}{geo}\PY{l+s+s1}{\PYZsq{}}\PY{p}{]}\PY{p}{[}\PY{l+s+s1}{\PYZsq{}}\PY{l+s+s1}{category}\PY{l+s+s1}{\PYZsq{}}\PY{p}{]}\PY{p}{[}\PY{l+s+s1}{\PYZsq{}}\PY{l+s+s1}{index}\PY{l+s+s1}{\PYZsq{}}\PY{p}{]}\PY{p}{:}
        \PY{k}{if} \PY{n}{label} \PY{o}{==} \PY{n}{index}\PY{p}{:}
            \PY{n}{countries} \PY{p}{[}\PY{n}{jsonEurostat}\PY{p}{[}\PY{l+s+s1}{\PYZsq{}}\PY{l+s+s1}{dimension}\PY{l+s+s1}{\PYZsq{}}\PY{p}{]}\PY{p}{[}\PY{l+s+s1}{\PYZsq{}}\PY{l+s+s1}{geo}\PY{l+s+s1}{\PYZsq{}}\PY{p}{]}\PY{p}{[}\PY{l+s+s1}{\PYZsq{}}\PY{l+s+s1}{category}\PY{l+s+s1}{\PYZsq{}}\PY{p}{]}\PY{p}{[}\PY{l+s+s1}{\PYZsq{}}\PY{l+s+s1}{index}\PY{l+s+s1}{\PYZsq{}}\PY{p}{]}\PY{p}{[}\PY{n}{index}\PY{p}{]}\PY{p}{]} \PY{o}{=} \PY{n}{jsonEurostat}\PY{p}{[}\PY{l+s+s1}{\PYZsq{}}\PY{l+s+s1}{dimension}\PY{l+s+s1}{\PYZsq{}}\PY{p}{]}\PY{p}{[}\PY{l+s+s1}{\PYZsq{}}\PY{l+s+s1}{geo}\PY{l+s+s1}{\PYZsq{}}\PY{p}{]}\PY{p}{[}\PY{l+s+s1}{\PYZsq{}}\PY{l+s+s1}{category}\PY{l+s+s1}{\PYZsq{}}\PY{p}{]}\PY{p}{[}\PY{l+s+s1}{\PYZsq{}}\PY{l+s+s1}{label}\PY{l+s+s1}{\PYZsq{}}\PY{p}{]}\PY{p}{[}\PY{n}{label}\PY{p}{]}\PY{o}{.}\PY{n}{replace}\PY{p}{(}\PY{l+s+s2}{\PYZdq{}}\PY{l+s+s2}{ (until 1990 former territory of the FRG)}\PY{l+s+s2}{\PYZdq{}}\PY{p}{,}\PY{l+s+s2}{\PYZdq{}}\PY{l+s+s2}{\PYZdq{}}\PY{p}{)}
            \PY{k}{break}

\PY{n}{values}\PY{o}{=} \PY{p}{[}\PY{p}{]} \PY{c+c1}{\PYZsh{}We will build an Array with the country names}
\PY{c+c1}{\PYZsh{} We start defining our output table, you can notice that we insert the geolocalitation and}
\PY{c+c1}{\PYZsh{} the geocoding values}
\PY{k}{def} \PY{n+nf}{camposTabla}\PY{p}{(}\PY{p}{)}\PY{p}{:} 
    \PY{n}{valor}\PY{p}{[}\PY{l+s+s1}{\PYZsq{}}\PY{l+s+s1}{Pais}\PY{l+s+s1}{\PYZsq{}}\PY{p}{]} \PY{o}{=} \PY{n}{countries}\PY{p}{[}\PY{n}{countryId}\PY{p}{]}
    \PY{n}{valor}\PY{p}{[}\PY{l+s+s1}{\PYZsq{}}\PY{l+s+s1}{Valor}\PY{l+s+s1}{\PYZsq{}}\PY{p}{]} \PY{o}{=} \PY{n}{jsonEurostat}\PY{p}{[}\PY{l+s+s1}{\PYZsq{}}\PY{l+s+s1}{value}\PY{l+s+s1}{\PYZsq{}}\PY{p}{]}\PY{p}{[}\PY{n}{dataId}\PY{p}{]}
\PY{c+c1}{\PYZsh{}     valor[\PYZsq{}numYear\PYZsq{}] = x}
    \PY{n}{valor}\PY{p}{[}\PY{l+s+s1}{\PYZsq{}}\PY{l+s+s1}{Year}\PY{l+s+s1}{\PYZsq{}}\PY{p}{]}\PY{o}{=} \PY{n}{year}
    \PY{n}{valor}\PY{p}{[}\PY{l+s+s1}{\PYZsq{}}\PY{l+s+s1}{Code}\PY{l+s+s1}{\PYZsq{}}\PY{p}{]}\PY{o}{=} \PY{n}{country\PYZus{}name\PYZus{}to\PYZus{}country\PYZus{}alpha2}\PY{p}{(}\PY{n}{countries}\PY{p}{[}\PY{n}{countryId}\PY{p}{]}\PY{p}{,} \PY{n}{cn\PYZus{}name\PYZus{}format}\PY{o}{=}\PY{l+s+s2}{\PYZdq{}}\PY{l+s+s2}{default}\PY{l+s+s2}{\PYZdq{}}\PY{p}{)}
    \PY{n}{valor}\PY{p}{[}\PY{l+s+s1}{\PYZsq{}}\PY{l+s+s1}{latitude}\PY{l+s+s1}{\PYZsq{}}\PY{p}{]}\PY{o}{=} \PY{n}{loc}\PY{o}{.}\PY{n}{latitude}
    \PY{n}{valor}\PY{p}{[}\PY{l+s+s1}{\PYZsq{}}\PY{l+s+s1}{longitude}\PY{l+s+s1}{\PYZsq{}}\PY{p}{]}\PY{o}{=} \PY{n}{loc}\PY{o}{.}\PY{n}{longitude}
    \PY{n}{values}\PY{o}{.}\PY{n}{append}\PY{p}{(}\PY{n}{valor}\PY{p}{)}
\PY{c+c1}{\PYZsh{} As we have 4 values for each country and in the JSPN the years have a ID between 0 and 3 }
\PY{c+c1}{\PYZsh{} 0:1990 , 1:2000, 2: 2010, 3:2020 we would need to group the values}
\PY{n}{year}\PY{o}{=} \PY{l+m+mi}{1990}
\PY{n}{x} \PY{o}{=} \PY{l+m+mi}{0}
\PY{k}{for} \PY{n}{dataId} \PY{o+ow}{in} \PY{n}{jsonEurostat}\PY{p}{[}\PY{l+s+s1}{\PYZsq{}}\PY{l+s+s1}{value}\PY{l+s+s1}{\PYZsq{}}\PY{p}{]}\PY{p}{:}
    \PY{n}{groupId} \PY{o}{=} \PY{p}{(}\PY{n+nb}{int}\PY{p}{(}\PY{n+nb}{int}\PY{p}{(}\PY{n}{dataId}\PY{p}{)}\PY{o}{/}\PY{l+m+mi}{4}\PY{p}{)}\PY{p}{)}
    \PY{k}{for} \PY{n}{countryId} \PY{o+ow}{in} \PY{n}{countries}\PY{p}{:}
        \PY{k}{if} \PY{n+nb}{str}\PY{p}{(}\PY{n}{groupId}\PY{p}{)} \PY{o}{==} \PY{n+nb}{str}\PY{p}{(}\PY{n}{countryId}\PY{p}{)}\PY{p}{:}
            \PY{n}{valor} \PY{o}{=} \PY{p}{\PYZob{}}\PY{p}{\PYZcb{}}
            \PY{n}{loc} \PY{o}{=}\PY{n}{geolocator}\PY{o}{.}\PY{n}{geocode}\PY{p}{(}\PY{n}{countries}\PY{p}{[}\PY{n}{countryId}\PY{p}{]}\PY{p}{)}
            \PY{k}{if} \PY{n}{x} \PY{o}{==} \PY{l+m+mi}{0}\PY{p}{:}
                \PY{n}{camposTabla}\PY{p}{(}\PY{p}{)}
                \PY{n}{x}\PY{o}{+}\PY{o}{=}\PY{l+m+mi}{1}
                \PY{n}{year} \PY{o}{=} \PY{l+m+mi}{2000}
                \PY{k}{break}
            \PY{k}{if} \PY{n}{x} \PY{o}{==} \PY{l+m+mi}{1}\PY{p}{:}
                \PY{n}{camposTabla}\PY{p}{(}\PY{p}{)}
                \PY{n}{x}\PY{o}{+}\PY{o}{=}\PY{l+m+mi}{1}
                \PY{n}{year} \PY{o}{=} \PY{l+m+mi}{2010}
                \PY{k}{break}
            \PY{k}{if} \PY{n}{x} \PY{o}{==} \PY{l+m+mi}{2}\PY{p}{:}
                \PY{n}{camposTabla}\PY{p}{(}\PY{p}{)}
                \PY{n}{x}\PY{o}{+}\PY{o}{=}\PY{l+m+mi}{1}
                \PY{n}{year} \PY{o}{=} \PY{l+m+mi}{2020}
                \PY{k}{break}
            \PY{k}{if} \PY{n}{x} \PY{o}{==} \PY{l+m+mi}{3}\PY{p}{:}
                \PY{n}{camposTabla}\PY{p}{(}\PY{p}{)}
                \PY{n}{x}\PY{o}{=}\PY{l+m+mi}{0}
                \PY{n}{year} \PY{o}{=} \PY{l+m+mi}{1990}
                \PY{k}{break}

\PY{c+c1}{\PYZsh{} values}
\PY{c+c1}{\PYZsh{} we can see how we ordered the data}
\end{Verbatim}
\end{tcolorbox}

    \begin{tcolorbox}[breakable, size=fbox, boxrule=1pt, pad at break*=1mm,colback=cellbackground, colframe=cellborder]
\prompt{In}{incolor}{5}{\boxspacing}
\begin{Verbatim}[commandchars=\\\{\}]
\PY{k+kn}{import} \PY{n+nn}{pandas} \PY{k}{as} \PY{n+nn}{pd}
\PY{c+c1}{\PYZsh{} pd.set\PYZus{}option(\PYZsq{}display.max\PYZus{}rows\PYZsq{}, 500)}
\PY{n}{df} \PY{o}{=} \PY{n}{pd}\PY{o}{.}\PY{n}{DataFrame}\PY{p}{(}\PY{n}{values}\PY{p}{)}
\PY{n}{year\PYZus{}1990}\PY{o}{=}\PY{n}{df}\PY{p}{[}\PY{l+s+s2}{\PYZdq{}}\PY{l+s+s2}{Year}\PY{l+s+s2}{\PYZdq{}}\PY{p}{]}\PY{o}{==}\PY{l+m+mi}{1990}
\PY{n}{filtro\PYZus{}year\PYZus{}1990}\PY{o}{=}\PY{n}{df}\PY{p}{[}\PY{n}{year\PYZus{}1990}\PY{p}{]}
\PY{n}{year\PYZus{}2000}\PY{o}{=}\PY{n}{df}\PY{p}{[}\PY{l+s+s2}{\PYZdq{}}\PY{l+s+s2}{Year}\PY{l+s+s2}{\PYZdq{}}\PY{p}{]}\PY{o}{==}\PY{l+m+mi}{2000}
\PY{n}{filtro\PYZus{}year\PYZus{}2000}\PY{o}{=}\PY{n}{df}\PY{p}{[}\PY{n}{year\PYZus{}2000}\PY{p}{]}
\PY{n}{year\PYZus{}2010}\PY{o}{=}\PY{n}{df}\PY{p}{[}\PY{l+s+s2}{\PYZdq{}}\PY{l+s+s2}{Year}\PY{l+s+s2}{\PYZdq{}}\PY{p}{]}\PY{o}{==}\PY{l+m+mi}{2010}
\PY{n}{filtro\PYZus{}year\PYZus{}2010}\PY{o}{=}\PY{n}{df}\PY{p}{[}\PY{n}{year\PYZus{}2010}\PY{p}{]}
\PY{n}{year\PYZus{}2020}\PY{o}{=}\PY{n}{df}\PY{p}{[}\PY{l+s+s2}{\PYZdq{}}\PY{l+s+s2}{Year}\PY{l+s+s2}{\PYZdq{}}\PY{p}{]}\PY{o}{==}\PY{l+m+mi}{2020}
\PY{n}{filtro\PYZus{}year\PYZus{}2020}\PY{o}{=}\PY{n}{df}\PY{p}{[}\PY{n}{year\PYZus{}2020}\PY{p}{]}

\PY{c+c1}{\PYZsh{} We can calculate the mean \PYZdq{}green\PYZdq{} surface}
\PY{n}{df2} \PY{o}{=} \PY{n}{df}\PY{o}{.}\PY{n}{groupby}\PY{p}{(}\PY{p}{[}\PY{l+s+s2}{\PYZdq{}}\PY{l+s+s2}{Pais}\PY{l+s+s2}{\PYZdq{}}\PY{p}{,}\PY{l+s+s2}{\PYZdq{}}\PY{l+s+s2}{Code}\PY{l+s+s2}{\PYZdq{}}\PY{p}{,}\PY{l+s+s2}{\PYZdq{}}\PY{l+s+s2}{latitude}\PY{l+s+s2}{\PYZdq{}}\PY{p}{,}\PY{l+s+s2}{\PYZdq{}}\PY{l+s+s2}{longitude}\PY{l+s+s2}{\PYZdq{}}\PY{p}{]}\PY{p}{)}\PY{o}{.}\PY{n}{Valor}\PY{o}{.}\PY{n}{mean}\PY{p}{(}\PY{p}{)}
\PY{n}{df2} \PY{o}{=} \PY{n}{df2}\PY{o}{.}\PY{n}{to\PYZus{}frame}\PY{p}{(}\PY{p}{)}
\end{Verbatim}
\end{tcolorbox}

    \begin{tcolorbox}[breakable, size=fbox, boxrule=1pt, pad at break*=1mm,colback=cellbackground, colframe=cellborder]
\prompt{In}{incolor}{6}{\boxspacing}
\begin{Verbatim}[commandchars=\\\{\}]
\PY{c+c1}{\PYZsh{} Create a world map to show distributions of users }
\PY{k+kn}{import} \PY{n+nn}{folium}
\PY{k+kn}{from} \PY{n+nn}{folium}\PY{n+nn}{.}\PY{n+nn}{plugins} \PY{k+kn}{import} \PY{n}{MarkerCluster}
\PY{k+kn}{import} \PY{n+nn}{branca}\PY{n+nn}{.}\PY{n+nn}{colormap} \PY{k}{as} \PY{n+nn}{cm}
\PY{c+c1}{\PYZsh{}empty map}
\PY{n}{linear} \PY{o}{=} \PY{n}{cm}\PY{o}{.}\PY{n}{LinearColormap}\PY{p}{(}\PY{p}{[}\PY{l+s+s2}{\PYZdq{}}\PY{l+s+s2}{green}\PY{l+s+s2}{\PYZdq{}}\PY{p}{,} \PY{l+s+s2}{\PYZdq{}}\PY{l+s+s2}{yellow}\PY{l+s+s2}{\PYZdq{}}\PY{p}{,} \PY{l+s+s2}{\PYZdq{}}\PY{l+s+s2}{red}\PY{l+s+s2}{\PYZdq{}}\PY{p}{]}\PY{p}{,} \PY{n}{vmin}\PY{o}{=}\PY{l+m+mi}{100}\PY{p}{,} \PY{n}{vmax}\PY{o}{=}\PY{l+m+mi}{100000}\PY{p}{)}

\PY{n}{world\PYZus{}map}\PY{o}{=} \PY{n}{folium}\PY{o}{.}\PY{n}{Map}\PY{p}{(}\PY{n}{tiles}\PY{o}{=}\PY{l+s+s2}{\PYZdq{}}\PY{l+s+s2}{cartodbpositron}\PY{l+s+s2}{\PYZdq{}}\PY{p}{)}
\PY{n}{marker\PYZus{}cluster} \PY{o}{=} \PY{n}{MarkerCluster}\PY{p}{(}\PY{p}{)}\PY{o}{.}\PY{n}{add\PYZus{}to}\PY{p}{(}\PY{n}{world\PYZus{}map}\PY{p}{)}
\PY{c+c1}{\PYZsh{}for each coordinate, create circlemarker of user percent}
\PY{k}{for} \PY{n}{i} \PY{o+ow}{in} \PY{n+nb}{range}\PY{p}{(}\PY{n+nb}{len}\PY{p}{(}\PY{n}{df}\PY{p}{)}\PY{p}{)}\PY{p}{:}
        \PY{n}{lat} \PY{o}{=} \PY{n}{df}\PY{o}{.}\PY{n}{iloc}\PY{p}{[}\PY{n}{i}\PY{p}{]}\PY{p}{[}\PY{l+s+s1}{\PYZsq{}}\PY{l+s+s1}{latitude}\PY{l+s+s1}{\PYZsq{}}\PY{p}{]}
        \PY{n}{long} \PY{o}{=} \PY{n}{df}\PY{o}{.}\PY{n}{iloc}\PY{p}{[}\PY{n}{i}\PY{p}{]}\PY{p}{[}\PY{l+s+s1}{\PYZsq{}}\PY{l+s+s1}{longitude}\PY{l+s+s1}{\PYZsq{}}\PY{p}{]}
        \PY{n}{radius}\PY{o}{=}\PY{l+m+mi}{10}
        \PY{n}{popup\PYZus{}text} \PY{o}{=} \PY{l+s+s2}{\PYZdq{}\PYZdq{}\PYZdq{}}\PY{l+s+s2}{\PYZlt{}h4\PYZgt{}Country Name :\PYZlt{}/h4\PYZgt{} }\PY{l+s+si}{\PYZob{}\PYZcb{}}\PY{l+s+s2}{\PYZlt{}br\PYZgt{}}
\PY{l+s+s2}{                    \PYZlt{}h4\PYZgt{}Area of Wooded Land :\PYZlt{}/h4\PYZgt{} }\PY{l+s+si}{\PYZob{}\PYZcb{}}\PY{l+s+s2}{\PYZlt{}br\PYZgt{}}
\PY{l+s+s2}{                    \PYZlt{}h4\PYZgt{}Year:\PYZlt{}/h4\PYZgt{} }\PY{l+s+si}{\PYZob{}\PYZcb{}}\PY{l+s+s2}{\PYZlt{}br\PYZgt{}}\PY{l+s+s2}{\PYZdq{}\PYZdq{}\PYZdq{}}
        \PY{n}{popup\PYZus{}text} \PY{o}{=} \PY{n}{popup\PYZus{}text}\PY{o}{.}\PY{n}{format}\PY{p}{(}\PY{n}{df}\PY{o}{.}\PY{n}{iloc}\PY{p}{[}\PY{n}{i}\PY{p}{]}\PY{p}{[}\PY{l+s+s1}{\PYZsq{}}\PY{l+s+s1}{Pais}\PY{l+s+s1}{\PYZsq{}}\PY{p}{]}\PY{p}{,}
                                   \PY{n}{df}\PY{o}{.}\PY{n}{iloc}\PY{p}{[}\PY{n}{i}\PY{p}{]}\PY{p}{[}\PY{l+s+s1}{\PYZsq{}}\PY{l+s+s1}{Valor}\PY{l+s+s1}{\PYZsq{}}\PY{p}{]}\PY{p}{,}
                                       \PY{n}{df}\PY{o}{.}\PY{n}{iloc}\PY{p}{[}\PY{n}{i}\PY{p}{]}\PY{p}{[}\PY{l+s+s1}{\PYZsq{}}\PY{l+s+s1}{Year}\PY{l+s+s1}{\PYZsq{}}\PY{p}{]}
                                   \PY{p}{)}
        \PY{n}{folium}\PY{o}{.}\PY{n}{CircleMarker}\PY{p}{(}\PY{n}{location} \PY{o}{=} \PY{p}{[}\PY{n}{lat}\PY{p}{,} \PY{n}{long}\PY{p}{]}\PY{p}{,} \PY{n}{radius}\PY{o}{=}\PY{n}{radius}\PY{p}{,} \PY{n}{popup}\PY{o}{=} \PY{n}{popup\PYZus{}text}\PY{p}{,} \PY{n}{fill} \PY{o}{=}\PY{k+kc}{True}\PY{p}{)}\PY{o}{.}\PY{n}{add\PYZus{}to}\PY{p}{(}\PY{n}{marker\PYZus{}cluster}\PY{p}{)}
\PY{c+c1}{\PYZsh{}show the map}
\PY{n}{world\PYZus{}map}
\end{Verbatim}
\end{tcolorbox}

            \begin{tcolorbox}[breakable, size=fbox, boxrule=.5pt, pad at break*=1mm, opacityfill=0]
\prompt{Out}{outcolor}{6}{\boxspacing}
\begin{Verbatim}[commandchars=\\\{\}]
<folium.folium.Map at 0x1a855777f40>
\end{Verbatim}
\end{tcolorbox}
        
    \begin{tcolorbox}[breakable, size=fbox, boxrule=1pt, pad at break*=1mm,colback=cellbackground, colframe=cellborder]
\prompt{In}{incolor}{7}{\boxspacing}
\begin{Verbatim}[commandchars=\\\{\}]
\PY{k+kn}{import} \PY{n+nn}{geopip}

\PY{n}{geopip}\PY{o}{.}\PY{n}{search}\PY{p}{(}\PY{n}{lng}\PY{o}{=}\PY{l+m+mf}{4.910248}\PY{p}{,} \PY{n}{lat}\PY{o}{=}\PY{l+m+mf}{50.850981}\PY{p}{)}
\end{Verbatim}
\end{tcolorbox}

            \begin{tcolorbox}[breakable, size=fbox, boxrule=.5pt, pad at break*=1mm, opacityfill=0]
\prompt{Out}{outcolor}{7}{\boxspacing}
\begin{Verbatim}[commandchars=\\\{\}]
\{'FIPS': 'BE',
 'ISO2': 'BE',
 'ISO3': 'BEL',
 'UN': 56,
 'NAME': 'Belgium',
 'AREA': 0,
 'POP2005': 10398049,
 'REGION': 150,
 'SUBREGION': 155,
 'LON': 4.664,
 'LAT': 50.643\}
\end{Verbatim}
\end{tcolorbox}
        
    \begin{tcolorbox}[breakable, size=fbox, boxrule=1pt, pad at break*=1mm,colback=cellbackground, colframe=cellborder]
\prompt{In}{incolor}{8}{\boxspacing}
\begin{Verbatim}[commandchars=\\\{\}]
\PY{k+kn}{from} \PY{n+nn}{shapely}\PY{n+nn}{.}\PY{n+nn}{geometry} \PY{k+kn}{import} \PY{n}{Point}

\PY{c+c1}{\PYZsh{} combine lat and lon column to a shapely Point() object}
\PY{n}{df}\PY{p}{[}\PY{l+s+s1}{\PYZsq{}}\PY{l+s+s1}{geometry}\PY{l+s+s1}{\PYZsq{}}\PY{p}{]} \PY{o}{=} \PY{n}{df}\PY{o}{.}\PY{n}{apply}\PY{p}{(}\PY{k}{lambda} \PY{n}{x}\PY{p}{:} \PY{n}{Point}\PY{p}{(}\PY{p}{(}\PY{n+nb}{float}\PY{p}{(}\PY{n}{x}\PY{o}{.}\PY{n}{longitude}\PY{p}{)}\PY{p}{,} \PY{n+nb}{float}\PY{p}{(}\PY{n}{x}\PY{o}{.}\PY{n}{latitude}\PY{p}{)}\PY{p}{)}\PY{p}{)}\PY{p}{,} \PY{n}{axis}\PY{o}{=}\PY{l+m+mi}{1}\PY{p}{)}
\end{Verbatim}
\end{tcolorbox}

    \begin{tcolorbox}[breakable, size=fbox, boxrule=1pt, pad at break*=1mm,colback=cellbackground, colframe=cellborder]
\prompt{In}{incolor}{9}{\boxspacing}
\begin{Verbatim}[commandchars=\\\{\}]
\PY{k+kn}{import} \PY{n+nn}{geopandas}
\PY{k+kn}{import} \PY{n+nn}{fiona}
\PY{n}{df} \PY{o}{=} \PY{n}{geopandas}\PY{o}{.}\PY{n}{GeoDataFrame}\PY{p}{(}\PY{n}{df}\PY{p}{,}\PY{n}{geometry}\PY{o}{=}\PY{l+s+s1}{\PYZsq{}}\PY{l+s+s1}{geometry}\PY{l+s+s1}{\PYZsq{}}\PY{p}{)} 
\PY{c+c1}{\PYZsh{} df.to\PYZus{}json(r\PYZsq{}C:\PYZbs{}Users\PYZbs{}gasca\PYZbs{}Desktop\PYZbs{}Export\PYZus{}geometry.json\PYZsq{})}
\PY{c+c1}{\PYZsh{} df.to\PYZus{}json(r\PYZsq{}C:\PYZbs{}Users\PYZbs{}gasca\PYZbs{}Desktop\PYZbs{}Export\PYZus{}geometry.json\PYZsq{})}
\PY{n}{df}
\PY{c+c1}{\PYZsh{} df.to\PYZus{}file(r\PYZsq{}C:\PYZbs{}Users\PYZbs{}gasca\PYZbs{}Desktop\PYZbs{}Export\PYZus{}geometry.geojson\PYZsq{}, driver=\PYZsq{}GeoJSON\PYZsq{})  }
\end{Verbatim}
\end{tcolorbox}

            \begin{tcolorbox}[breakable, size=fbox, boxrule=.5pt, pad at break*=1mm, opacityfill=0]
\prompt{Out}{outcolor}{9}{\boxspacing}
\begin{Verbatim}[commandchars=\\\{\}]
                       Pais    Valor  Year Code   latitude  longitude  \textbackslash{}
0                   Austria  3775.67  1990   AT  47.200000  13.200000
1                   Austria  3838.14  2000   AT  47.200000  13.200000
2                   Austria  3863.20  2010   AT  47.200000  13.200000
3                   Austria  3899.15  2020   AT  47.200000  13.200000
4    Bosnia and Herzegovina  2210.00  1990   BA  44.305348  17.596147
..                      {\ldots}      {\ldots}   {\ldots}  {\ldots}        {\ldots}        {\ldots}
119                Slovakia  1925.90  2020   SK  48.741152  19.452865
120          United Kingdom  2778.00  1990   GB  54.702354  -3.276575
121          United Kingdom  2954.00  2000   GB  54.702354  -3.276575
122          United Kingdom  3059.00  2010   GB  54.702354  -3.276575
123          United Kingdom  3190.00  2020   GB  54.702354  -3.276575

                      geometry
0    POINT (13.20000 47.20000)
1    POINT (13.20000 47.20000)
2    POINT (13.20000 47.20000)
3    POINT (13.20000 47.20000)
4    POINT (17.59615 44.30535)
..                         {\ldots}
119  POINT (19.45286 48.74115)
120  POINT (-3.27658 54.70235)
121  POINT (-3.27658 54.70235)
122  POINT (-3.27658 54.70235)
123  POINT (-3.27658 54.70235)

[124 rows x 7 columns]
\end{Verbatim}
\end{tcolorbox}
        

    % Add a bibliography block to the postdoc
    
    
    
\end{document}
